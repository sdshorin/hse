\documentclass[a4paper]{article}


\usepackage[T2A]{fontenc}
\usepackage[utf8]{inputenc}
\usepackage[english,russian]{babel}


\usepackage{amsmath, amsfonts, amssymb, amsthm, mathtools}

\author{Шорин Сергей, БКНАД211}
\title{Линейная Алгебра ДЗ 3}
\date{\today}

\begin{document}

\maketitle

\newpage

\section*{1.a}

Если матрицу 3x4 умножить справа на
\begin{equation*}
\begin{pmatrix}
1 & 0 & 3 & 0\\
0 & 1 & 0 & 0\\
0 & 0 & 1 & 0\\
0 & 0 & 0 & 1\\
\end{pmatrix}
\end{equation*}, то первый столбец, умноженный на 3, прибавится к 3 столбцу.



\section*{1.б}
Если матрицу 3x4 умножить слева на
\begin{equation*}
\begin{pmatrix}
1 & 0 & 0\\
-4 & 1 & 0\\
0 & 0 & 1\\
\end{pmatrix}
\end{equation*}, то вторая строка, умноженная на 4, будет вычтена из второй строки.



\section*{1.в}

Если матрицу 3x4 умножить справа на
\begin{equation*}
\begin{pmatrix}
0 & 0 & 0 & 1\\
0 & 1 & 0 & 0\\
0 & 0 & 1 & 0\\
1 & 0 & 0 & 0\\
\end{pmatrix}
\end{equation*}, то первый столбец поменяется местами с четвертым столбцом.



\section*{1.г}
Если матрицу 3x4 умножить слева на
\begin{equation*}
\begin{pmatrix}
1 & 0 & 0\\
0 & -2 & 0\\
0 & 0 & 1\\
\end{pmatrix}
\end{equation*}, то вторая строка умножится на -2.



\section*{2.а}
Чтобы вторая строка поменялась с четвертой, нужно домножить матрицу A слева на матрицу 
\begin{equation*}
\begin{pmatrix}
1 & 0 & 0 & 0\\
0 & 0 & 0 & 1\\
0 & 0 & 1 & 0\\
0 & 1 & 0 & 0\\
\end{pmatrix}
\end{equation*}




\section*{2.б}
Чтобы третий столбец домножить на 2, нужно домножить матрицу A справа на матрицу 
\begin{equation*}
\begin{pmatrix}
1 & 0 & 0\\
0 & 1 & 0\\
0 & 0 & 2\\
\end{pmatrix}
\end{equation*}

\section*{2.в}
Чтобы прибавить третью строку к первой с коэффициентом -5, нужно домножить матрицу A слева на матрицу 
\begin{equation*}
\begin{pmatrix}
1 & 0 & -5 & 0\\
0 & 1 & 0 & 0\\
0 & 0 & 1 & 0\\
0 & 0 & 0 & 1\\
\end{pmatrix}
\end{equation*}

\section*{2.г}
Чтобы прибавить второй столбец к первому с коэффициентом 15, нужно домножить матрицу A справа на матрицу 
\begin{equation*}
\begin{pmatrix}
1 & 0 & 0\\
15 & 1 & 0\\
0 & 0 & 2\\
\end{pmatrix}
\end{equation*}

\section*{3.а}
Найти обратную матрицу к 
\begin{equation*}
\begin{pmatrix}
1 & 3\\
4 & 11\\
\end{pmatrix}
\end{equation*}

\begin{equation*}
 \left(\begin{array}{rr|rr}
   1 & 3 & 1 & 0\\
   4 & 11 & 0 & 1\\
   \end{array}\right)_{II -= I * 4}
   \approx
   \left(\begin{array}{rr|rr}
   1 & 3 & 1 & 0\\
   0 & -1 & -4 & 1\\
   \end{array}\right)
\approx
   \left(\begin{array}{rr|rr}
   1 & 3 & 1 & 0\\
   0 & 1 & 4 & -1\\
   \end{array}\right)
  \approx
\end{equation*}

\begin{equation*}
\approx
   \left(\begin{array}{rr|rr}
   1 & 0 & -11 & 3\\
   0 & 1 & 4 & -1\\
   \end{array}\right)
  \approx
\end{equation*}

\subsection{Ответ:}
\begin{equation*}
A^{-1} = \begin{pmatrix}
-11 & 3\\
4 & -1\\
\end{pmatrix}
\end{equation*}



\section*{3.б}
Найти обратную матрицу к 
\begin{equation*}
\begin{pmatrix}
3 & -4 & 5\\
2 & -3 & 1\\
3 & -5 & -1\\
\end{pmatrix}
\end{equation*}

\begin{equation*}
 \left(\begin{array}{rrr|rrr}
  3 & -4 & 5 & 1 & 0 & 0\\
2 & -3 & 1& 0 & 1 & 0\\
3 & -5 & -1 & 0 & 0 & 1\\
   \end{array}\right)
   \approx
 \left(\begin{array}{rrr|rrr}
  1 & -1 & 4 & 1 & -1 & 0\\
2 & -3 & 1& 0 & 1 & 0\\
1 & -2 & -2 & 0 & -1 & 1\\
   \end{array}\right)
\approx
 \left(\begin{array}{rrr|rrr}
  1 & -1 & 4 & 1 & -1 & 0\\
0 & -1 & -7& -2 & 3 & 0\\
0 & -1 & -6 & -1 & 0 & 1\\
   \end{array}\right)
  \approx
\end{equation*}
\begin{equation*}
 \left(\begin{array}{rrr|rrr}
  1 & -1 & 4 & 1 & -1 & 0\\
0 & 1 & 7 & 2 & -3 & 0\\
0 & 0 & 1 & 1 & -3 & 1\\
   \end{array}\right)
   \approx
 \left(\begin{array}{rrr|rrr}
1 & 0 & 11 & 3 & -4 & 0\\
0 & 1 & 7 & 2 & -3 & 0\\
0 & 0 & 1 & 1 & -3 & 1\\
   \end{array}\right)
\approx
 \left(\begin{array}{rrr|rrr}
1 & 0 & 0 & -9 & 29 & -11\\
0 & 1 & 0 & -5 & 18 & -7\\
0 & 0 & 1 & 1 & -3 & 1\\
   \end{array}\right)
  \approx
\end{equation*}


\subsection{Ответ:}
\begin{equation*}
A^{-1} = \begin{pmatrix}
-9 & 29 & -11\\
-5 & 18 & -7\\
1 & -3 & 1\\
\end{pmatrix}
\end{equation*}




\section*{3.в}
Найти обратную матрицу к 
\begin{equation*}
\begin{pmatrix}
2 & 3 & 1 & 2\\
1 & 1 & 2 & 0\\
0 & 0 & 1 & -1\\
0 & 0 & 1 & -2\\
\end{pmatrix}
\end{equation*}

\begin{equation*}
 \left(\begin{array}{rrrr|rrrr}
 2 & 3 & 1 & 2 & 1 & 0 & 0 & 0\\
1 & 1 & 2 & 0 & 0 & 1 & 0 & 0\\
0 & 0 & 1 & -1 & 0 & 0 & 1 & 0\\
0 & 0 & 1 & -2 & 0 & 0 & 0 & 1\\
   \end{array}\right)
   \approx
 \left(\begin{array}{rrrr|rrrr}
 2 & 3 & 1 & 2 & 1 & 0 & 0 & 0\\
1 & 1 & 2 & 0 & 0 & 1 & 0 & 0\\
0 & 0 & 1 & -1 & 0 & 0 & 1 & 0\\
0 & 0 & 0 & -1 & 0 & 0 & -1 & 1\\
   \end{array}\right)
\approx
\end{equation*}
\begin{equation*}
  \left(\begin{array}{rrrr|rrrr}
 2 & 3 & 1 & 2 & 1 & 0 & 0 & 0\\
1 & 1 & 2 & 0 & 0 & 1 & 0 & 0\\
0 & 0 & 1 & 0 & 0 & 0 & 2 & -1\\
0 & 0 & 0 & 1 & 0 & 0 & 1 & -1\\
   \end{array}\right)
   \approx
  \left(\begin{array}{rrrr|rrrr}
2 & 3 & 1 & 0 & 1 & 0 & -2 & 2\\
1 & 1 & 2 & 0 & 0 & 1 & 0 & 0\\
0 & 0 & 1 & 0 & 0 & 0 & 2 & -1\\
0 & 0 & 0 & 1 & 0 & 0 & 1 & -1\\
   \end{array}\right)
\approx
\end{equation*}
\begin{equation*}
  \left(\begin{array}{rrrr|rrrr}
1 & 1 & 0 & 0 & 0 & 1 & -4 & 2\\
2 & 3 & 0 & 0 & 1 & 0 & -4 & 3\\
0 & 0 & 1 & 0 & 0 & 0 & 2 & -1\\
0 & 0 & 0 & 1 & 0 & 0 & 1 & -1\\
   \end{array}\right)
   \approx
  \left(\begin{array}{rrrr|rrrr}
1 & 1 & 0 & 0 & 0 & 1 & -4 & 2\\
0 & 1 & 0 & 0 & 1 & -2 & 4 & -1\\
0 & 0 & 1 & 0 & 0 & 0 & 2 & -1\\
0 & 0 & 0 & 1 & 0 & 0 & 1 & -1\\
   \end{array}\right)
\approx
\end{equation*}
\begin{equation*}
\approx
  \left(\begin{array}{rrrr|rrrr}
1 & 0 & 0 & 0 & -1 & 3 & -8 & 3\\
0 & 1 & 0 & 0 & 1 & -2 & 4 & -1\\
0 & 0 & 1 & 0 & 0 & 0 & 2 & -1\\
0 & 0 & 0 & 1 & 0 & 0 & 1 & -1\\
   \end{array}\right)
\end{equation*}

\subsection{Ответ:}
\begin{equation*}
A^{-1} = \begin{pmatrix}
-1 & 3 & -8 & 3\\
 1 & -2 & 4 & -1\\
0 & 0 & 2 & -1\\
0 & 0 & 1 & -1\\
\end{pmatrix}
\end{equation*}


\section*{4.a}
Представьте в виде произведения матриц элементарных преобразований:
\begin{equation*}
\begin{pmatrix}
1 & 2\\
3 & 7\\
\end{pmatrix}
\end{equation*}

\begin{equation*}
\begin{pmatrix}
1 & 0\\
3 & 1\\
\end{pmatrix}
\begin{pmatrix}
1 & 2\\
0 & 1\\
\end{pmatrix} = 
\begin{pmatrix}
1 & 2\\
3 & 7\\
\end{pmatrix}
\end{equation*}
\begin{equation*}
\begin{pmatrix}
1 & 0\\
3 & 1\\
\end{pmatrix}
\begin{pmatrix}
1 & 2\\
0 & 1\\
\end{pmatrix}
\begin{pmatrix}
1 & 0\\
0 & 1\\
\end{pmatrix} = 
\begin{pmatrix}
1 & 2\\
3 & 7\\
\end{pmatrix}
\end{equation*}
\subsection*{Ответ:}
\begin{equation*}
\begin{pmatrix}
1 & 0\\
3 & 1\\
\end{pmatrix}
\begin{pmatrix}
1 & 2\\
0 & 1\\
\end{pmatrix}
\end{equation*}



\section*{4.б}
Представьте в виде произведения матриц элементарных преобразований:
\begin{equation*}
\begin{pmatrix}
2 & 0 & 0\\
3 & 1 & 1\\
0 & 0 & 2\\
\end{pmatrix}
\end{equation*}

\begin{equation*}
\begin{pmatrix}
2 & 0 & 0\\
0 & 1 & 0\\
0 & 0 & 2
\end{pmatrix}
\begin{pmatrix}
1 & 0 & 0\\
3 & 1 & 1\\
0 & 0 & 1\\
\end{pmatrix} = 
\begin{pmatrix}
2 & 0 & 0\\
3 & 1 & 1\\
0 & 0 & 2\\
\end{pmatrix}
\end{equation*}
\begin{equation*}
\begin{pmatrix}
2 & 0 & 0\\
0 & 1 & 0\\
0 & 0 & 2
\end{pmatrix}
\begin{pmatrix}
1 & 0 & 0\\
0 & 1 & 1\\
0 & 0 & 1
\end{pmatrix}
\begin{pmatrix}
1 & 0 & 0\\
3 & 1 & 0\\
0 & 0 & 1\\
\end{pmatrix} = 
\begin{pmatrix}
2 & 0 & 0\\
3 & 1 & 1\\
0 & 0 & 2\\
\end{pmatrix}
\end{equation*}
\begin{equation*}
\begin{pmatrix}
2 & 0 & 0\\
0 & 1 & 0\\
0 & 0 & 2
\end{pmatrix}
\begin{pmatrix}
1 & 0 & 0\\
0 & 1 & 1\\
0 & 0 & 1
\end{pmatrix}
\begin{pmatrix}
1 & 0 & 0\\
3 & 1 & 0\\
0 & 0 & 1\\
\end{pmatrix}
\begin{pmatrix}
1 & 0 & 0\\
0 & 1 & 0\\
0 & 0 & 1\\
\end{pmatrix} = 
\begin{pmatrix}
2 & 0 & 0\\
3 & 1 & 1\\
0 & 0 & 2\\
\end{pmatrix}
\end{equation*}
\subsection*{Ответ:}
\begin{equation*}
\begin{pmatrix}
2 & 0 & 0\\
0 & 1 & 0\\
0 & 0 & 2
\end{pmatrix}
\begin{pmatrix}
1 & 0 & 0\\
0 & 1 & 1\\
0 & 0 & 1
\end{pmatrix}
\begin{pmatrix}
1 & 0 & 0\\
3 & 1 & 0\\
0 & 0 & 1\\
\end{pmatrix}
\end{equation*}

\section*{4.a}

Решим матричное уравнение:
\begin{equation*}
X\begin{pmatrix}
-1 & 1 \\
3 & -4
\end{pmatrix}
= 
\begin{pmatrix}
-2 & -1 \\
3 & 4
\end{pmatrix}
\end{equation*}
Найдем обратную матрицу:

\begin{equation*}
 \left(\begin{array}{rr|rr}
   -1 & 1 & 1 & 0\\
   3 & -4 & 0 & 1\\
   \end{array}\right)
   \approx
   \left(\begin{array}{rr|rr}
   -1 & 1 & 1 & 0\\
   0 & -1 & 3 & 1\\
   \end{array}\right)
\approx
   \left(\begin{array}{rr|rr}
   -1 & 0 & 4 & 1\\
   0 & -1 & 3 & 1\\
   \end{array}\right)
  \approx
      \left(\begin{array}{rr|rr}
   1 & 0 & -4 & -1\\
   0 & 1 & -3 & -1\\
   \end{array}\right)
\end{equation*}

Домножим равенство на эту матрицу справа:
\begin{equation*}
X\begin{pmatrix}
-1 & 1 \\
3 & -4
\end{pmatrix}
\begin{pmatrix}
-4 & -1 \\
-3 & -1
\end{pmatrix}
= 
\begin{pmatrix}
-2 & -1 \\
3 & 4
\end{pmatrix}
\begin{pmatrix}
-4 & -1 \\
-3 & -1
\end{pmatrix} = X
\end{equation*}
\begin{equation*}
\begin{pmatrix}
-2 & -1 \\
3 & 4
\end{pmatrix}
\begin{pmatrix}
-4 & -1 \\
-3 & -1
\end{pmatrix} =
\begin{pmatrix}
11 & 3 \\
-24 & -7
\end{pmatrix}
= X
\end{equation*}
\subsection{Ответ:}
\begin{equation*}
X = \begin{pmatrix}
11 & 3 \\
-24 & -7
\end{pmatrix}
\end{equation*}


\section*{4.б}

Решим матричное уравнение:
\begin{equation*}
\begin{pmatrix}
1 & 2 & -3 \\
3 & 2 & -4 \\
2 & -1 & 0 \\
\end{pmatrix}
X = \begin{pmatrix}
1 & -3 & 0 \\
10 & 2 & 7 \\
10 & 7 & 8 \\
\end{pmatrix}
\end{equation*}

Найдем обратную матрицу:

\begin{equation*}
 \left(\begin{array}{rrr|rrr}
  1 & 2 & -3 & 1 & 0 & 0\\
3 & 2 & -4& 0 & 1 & 0\\
2 & -1 & 0 & 0 & 0 & 1\\
   \end{array}\right)
   \approx
 \left(\begin{array}{rrr|rrr}
1 & 2 & -3 & 1 & 0 & 0\\
0 & -4 & 5& -3 & 1 & 0\\
0 & -5 & 6 & -2 & 0 & 1\\
   \end{array}\right)
\approx
 \left(\begin{array}{rrr|rrr}
1 & 2 & -3 & 1 & 0 & 0\\
0 & -4 & 5& -3 & 1 & 0\\
0 & -1 & 1 & 1 & -1 & 1\\
   \end{array}\right)
  \approx
\end{equation*}

\begin{equation*}
 \left(\begin{array}{rrr|rrr}
1 & 2 & -3 & 1 & 0 & 0\\
0 & 0 & 1& -7 & 5 & -4\\
0 & -1 & 1 & 1 & -1 & 1\\
   \end{array}\right)
   \approx
 \left(\begin{array}{rrr|rrr}
1 & 2 & -3 & 1 & 0 & 0\\
0 & -1 & 1 & 1 & -1 & 1\\
0 & 0 & 1& -7 & 5 & -4\\
   \end{array}\right)
\approx
 \left(\begin{array}{rrr|rrr}
1 & 2 & -3 & 1 & 0 & 0\\
0 & -1 & 0 & 8 & -6 & 5\\
0 & 0 & 1& -7 & 5 & -4\\
   \end{array}\right)
  \approx
\end{equation*}


\begin{equation*}
 \left(\begin{array}{rrr|rrr}
1 & 2 & -3 & 1 & 0 & 0\\
0 & 1 & 0 & -8 & 6 & -5\\
0 & 0 & 1& -7 & 5 & -4\\
   \end{array}\right)
   \approx
 \left(\begin{array}{rrr|rrr}
1 & 2 & 0 & -20 & 15 & -12\\
0 & 1 & 0 & -8 & 6 & -5\\
0 & 0 & 1& -7 & 5 & -4\\
   \end{array}\right)
\approx
 \left(\begin{array}{rrr|rrr}
1 & 0 & 0 & -4 & 3 & -2\\
0 & 1 & 0 & -8 & 6 & -5\\
0 & 0 & 1& -7 & 5 & -4\\
   \end{array}\right)
\end{equation*}

Домножим равенство слева на эту матрицу:
\begin{equation*}
\begin{pmatrix}
-4 & 3 & -2 \\
-8 & 6 & -5 \\
-7 & 5 & -4 \\
\end{pmatrix}
\begin{pmatrix}
1 & 2 & -3 \\
3 & 2 & -4 \\
2 & -1 & 0 \\
\end{pmatrix}
X = 
\begin{pmatrix}
-4 & 3 & -2 \\
-8 & 6 & -5 \\
-7 & 5 & -4 \\
\end{pmatrix}
\begin{pmatrix}
1 & -3 & 0 \\
10 & 2 & 7 \\
10 & 7 & 8 \\
\end{pmatrix}
\end{equation*}
\begin{equation*}
\begin{pmatrix}
1 & 0 & 0 \\
0 & 1 & 0 \\
0 & 0 & 1 \\
\end{pmatrix}
X = 
\begin{pmatrix}
-4 & 3 & -2 \\
-8 & 6 & -5 \\
-7 & 5 & -4 \\
\end{pmatrix}
\begin{pmatrix}
1 & -3 & 0 \\
10 & 2 & 7 \\
10 & 7 & 8 \\
\end{pmatrix} = 
\begin{pmatrix}
6 & 4 & 5 \\
2 & 1 & 2 \\
3 & 3 & 3 \\
\end{pmatrix}
\end{equation*}

\subsection*{Ответ:}
\begin{equation*}
X = 
\begin{pmatrix}
6 & 4 & 5 \\
2 & 1 & 2 \\
3 & 3 & 3 \\
\end{pmatrix}
\end{equation*}

\end{document}