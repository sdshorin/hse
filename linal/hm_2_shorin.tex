\documentclass[a4paper]{article}


\usepackage[T2A]{fontenc}
\usepackage[utf8]{inputenc}
\usepackage[english,russian]{babel}


\usepackage{amsmath, amsfonts, amssymb, amsthm, mathtools}

\author{Шорин Сергей, БКНАД211}
\title{Линейная Алгебра ДЗ 2}
\date{\today}

\begin{document}

\maketitle

\newpage

\section*{1}
Вычислим выражение:

\begin{equation*}
3
\begin{pmatrix}
1 &2\\
4&3\\
1&1\\
1&-1\\
\end{pmatrix} +
\begin{pmatrix}
  1&0&1\\
1&2&0\\
2&1&0\\
1&3&2\\
\end{pmatrix}
\begin{pmatrix}
 1&0\\
0&1\\
0&0\\
\end{pmatrix}
-
\begin{pmatrix}
 1\\
2\\
3\\
4\\
\end{pmatrix}
\begin{pmatrix}
1&2
\end{pmatrix}
\end{equation*}
разобьём на части: \\
А)
\begin{equation*}
3
\begin{pmatrix}
1 &2\\
4&3\\
1&1\\
1&-1\\
\end{pmatrix}  = 
\begin{pmatrix}
3 &6\\
12&9\\
3&3\\
3&-3\\
\end{pmatrix}
\end{equation*}
Б)
\begin{equation*}
\begin{pmatrix}
  1&0&1\\
1&2&0\\
2&1&0\\
1&3&2\\
\end{pmatrix}
\begin{pmatrix}
 1&0\\
0&1\\
0&0\\
\end{pmatrix} =
\end{equation*} 


\begin{equation*}
=
\begin{pmatrix}
 1*1 +0*0 +1*0& 1*0 +0*1 +1*0\\
1*1 +2 *0 +0*0&1*0 +2*1 +0*0\\
2*1 +1*0 +0*0&2*0 +1*1 +0*0\\
1*1 +3*0 +2*0&1*0 +3*1 +2*0\\
\end{pmatrix}=
\begin{pmatrix}
  1&0\\
1&2\\
2&1\\
1&3\\
\end{pmatrix}
\end{equation*} 


В)

\begin{equation*}
\begin{pmatrix}
 1\\
2\\
3\\
4\\
\end{pmatrix}
\begin{pmatrix}
1&2
\end{pmatrix} = 
\begin{pmatrix}
1 &2\\
2&4\\
3&6\\
4&8\\
\end{pmatrix}
\end{equation*}
Итого:

\begin{equation*}
\begin{pmatrix}
3 &6\\
12&9\\
3&3\\
3&-3\\
\end{pmatrix}+
\begin{pmatrix}
  1&0\\
1&2\\
2&1\\
1&3\\
\end{pmatrix}-
\begin{pmatrix}
1 &2\\
2&4\\
3&6\\
4&8\\
\end{pmatrix}=
\end{equation*}


\begin{equation*}=
\begin{pmatrix}
3+1-1 &6+0-2\\
12+1-2&9+2-4\\
3+2-3&3+1-6\\
3+1-4&-3+3-8\\
\end{pmatrix}=
\begin{pmatrix}
3 &4\\
11&7\\
2&-2\\
0&-8\\
\end{pmatrix}
\end{equation*}
\subsection*{Ответ:}
\begin{equation*}
\begin{pmatrix}
3 &4\\
11&7\\
2&-2\\
0&-8\\
\end{pmatrix}
\end{equation*}
















\section*{2}
Вычислим максимально рациональным способом:\\
 tr((A + B)(X + Y) - XA - XB - YB)  =\\
= tr( AX + AY + BX + BY - XA - XB - YB) = \\
= tr(AX) + tr(AY) + tr(BX) + tr(BY) - tr(XA) - tr(XB) - tr(YB) = \\
= tr(AX) + tr(AY) + tr(BX) + tr(BY) - tr(AX) - tr(BX) - tr(BY) = \\
= tr(AY)  
\begin{equation*}
A = 
\begin{pmatrix}
1 &2&1\\
0&1&0\\
-1&0&1\\
\end{pmatrix}    Y = 
\begin{pmatrix}
1 &2&3\\
4&1&1\\
3&2&1\\
\end{pmatrix}
\end{equation*}

\begin{equation*}
AY = 
\begin{pmatrix}
1 &2&1\\
0&1&0\\
-1&0&1\\
\end{pmatrix}  
\begin{pmatrix}
1 &2&3\\
4&1&1\\
3&2&1\\
\end{pmatrix} = 
\begin{pmatrix}
12 &6&6\\
4&1&1\\
2&0&-2\\
\end{pmatrix}   
\end{equation*}

\begin{equation*}
tr(AY) = tr
\begin{pmatrix}
12 &6&6\\
4&1&1\\
2&0&-2\\
\end{pmatrix}   = 12 +1 -2 = 11
\end{equation*}
\subsection*{Ответ: 11}


\section*{3}
Найдем все матрицы, перестановочные с матрицей A:
\begin{equation*}
A = 
\begin{pmatrix}
1 &1&0\\
0&1&1\\
0&0&1\\
\end{pmatrix}
\end{equation*}

\begin{equation*}
AX = 
\begin{pmatrix}
1 &1&0\\
0&1&1\\
0&0&1\\
\end{pmatrix}
\begin{pmatrix}
a_{11} &a_{12}&a_{13}\\
a_{21}&a_{22}&a_{23}\\
a_{31}&a_{32}&a_{33}\\
\end{pmatrix} =
XA = 
\begin{pmatrix}
a_{11} &a_{12}&a_{13}\\
a_{21}&a_{22}&a_{23}\\
a_{31}&a_{32}&a_{33}\\
\end{pmatrix}
\begin{pmatrix}
1 &1&0\\
0&1&1\\
0&0&1\\
\end{pmatrix}
\end{equation*}
\begin{equation*}
\begin{pmatrix}
a_{11} + a_{21}&a_{12} + a_{22}&a_{13} + a_{23}\\
a_{21} + a_{31}&a_{22} + a_{32}&a_{23} + a_{33}\\
a_{31}&a_{32}&a_{33}\\
\end{pmatrix} = 
\begin{pmatrix}
a_{11} & a_{11} + a_{12}& a_{12} + a_{13}\\
a_{21}&a_{22} + a_{21}&a_{23} + a_{22}\\
a_{31}&a_{32} + a_{31}&a_{33} + a_{32}\\
\end{pmatrix}
\end{equation*}
\begin{equation*}
\begin{cases}
a_{21} = 0\\
a_{11} = a_{22}\\
a_{23} = a_{12}\\
a_{31} = 0\\
a_{32} = a_{21}\\
a_{33} = a_{22}\\
a_{31} = 0\\
a_{32} = 0\\
\end{cases}
=
\begin{cases}
a_{21} = 0\\
a_{11} = a_{22} = a_{33}\\
a_{23} = a_{12}\\
a_{31} = 0\\
a_{21} = 0\\
a_{31} = 0\\
a_{32} = 0\\
\end{cases}
\end{equation*}
\subsection*{Ответ}
\begin{equation*}
X = 
\begin{pmatrix}
a &b&0\\
0&a&b\\
0&0&a\\
\end{pmatrix}
\end{equation*}


\section*{4}
При каких $\lambda$ найдутся матрицы X, Y, такие что XY - YX = $\lambda E$?
$$XY - YX = \lambda E$$
Посмотрим на след матрицы $\lambda E$: по определению след единичной матрицы размера n равен n. $tr(\lambda E) = \lambda n$ (по свойству следа).
Рассмотрим след выражения
$$tr(XY - YX)$$
$$tr(XY) - tr(YX)$$
$$tr(XY) - tr(XY)$$
$$tr(XY) - tr(XY) = 0$$
Следовательно, $0 = \lambda n, n \neq 0$. Отсюда $\lambda = 0$
\subsection*{Ответ}
$$\lambda = 0$$

\section*{5}
Найдите все матрицы A размера $nxn$, что для любой матрицы B того же размера выполнено $AB = BA$.

Рассмотрим первый столбец и первую сточку такой матрицы. Пусть строка будет (a, b, c, d, ...), а столбец - (a, x, y, z, ...).

Обозначим строки и столбцы матрицы B как $(a_{11}, a_{12}, a_{13},...)$ и $(a_{11}, a_{21}, a_{31},...)$

Тогда первый элемент матрицы AB (или BA) должен быть равен $a_{11}a + a_{12}b +  a_{13}c + ... = a_{11}a + a_{21}x+ a_{31}y + ...$

Так как матрица B может быть любая, все ее элементы не зависят друг от друга - например, на место $x_{21}$ можно поставить элемент $x_{21} + 1$. Следовательно, элементы матрицы A кроме (a) в этих столбцах и строчках будут равняться 0.
Получается матрица вида 

\begin{equation*}
A = 
\begin{pmatrix}
a &0&0\\
0&b&0\\
0&0&c\\
\end{pmatrix}
\end{equation*}
Но при умножении на матрицу B слева элемент (c) будет увеличивать последний столбец матрицы B в (c) раз, а при умножении права - будет увеличивать последнюю строчку в (с) раз. Таким образом матрица AB будет равна BA только если все элементы на диагонали равны (последняя строчка умноженная в (c) раз - это то же самое, что последние элементы всех столбцов, умноженных на (c).

\subsection*{Ответ}
\begin{equation*}
A = 
\begin{pmatrix}
a &0&0\\
0&a&0\\
0&0&a\\
\end{pmatrix}
\end{equation*}


\end{document}