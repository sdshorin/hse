\documentclass[a4paper]{article}


\usepackage[T2A]{fontenc}
\usepackage[utf8]{inputenc}
\usepackage[english,russian]{babel}


\usepackage{amsmath, amsfonts, amssymb, amsthm, mathtools}

\author{Шорин Сергей, БКНАД211}
\title{Линейная Алгебра ДЗ 1}
\date{\today}

\begin{document}

\maketitle

\newpage

\section*{1}
Вычислим выражение:

\begin{equation*}
3
\begin{pmatrix}
1 &2\\
4&3\\
1&1\\
1&-1\\
\end{pmatrix} +
\begin{pmatrix}
  1&0&1\\
1&2&0\\
2&1&0\\
1&3&2\\
\end{pmatrix}
\begin{pmatrix}
 1&0\\
0&1\\
0&0\\
\end{pmatrix}
-
\begin{pmatrix}
 1\\
2\\
3\\
4\\
\end{pmatrix}
\begin{pmatrix}
1&2
\end{pmatrix}
\end{equation*}
разобьём на части: \\
А)
\begin{equation*}
3
\begin{pmatrix}
1 &2\\
4&3\\
1&1\\
1&-1\\
\end{pmatrix}  = 
\begin{pmatrix}
3 &6\\
12&9\\
3&3\\
3&-3\\
\end{pmatrix}
\end{equation*}
Б)
\begin{equation*}
\begin{pmatrix}
  1&0&1\\
1&2&0\\
2&1&0\\
1&3&2\\
\end{pmatrix}
\begin{pmatrix}
 1&0\\
0&1\\
0&0\\
\end{pmatrix} =
\end{equation*} 


\begin{equation*}
=
\begin{pmatrix}
 1*1 +0*0 +1*0& 1*0 +0*1 +1*0\\
1*1 +2 *0 +0*0&1*0 +2*1 +0*0\\
2*1 +1*0 +0*0&2*0 +1*1 +0*0\\
1*1 +3*0 +2*0&1*0 +3*1 +2*0\\
\end{pmatrix}=
\begin{pmatrix}
  1&0\\
1&2\\
2&1\\
1&3\\
\end{pmatrix}
\end{equation*} 


В)

\begin{equation*}
\begin{pmatrix}
 1\\
2\\
3\\
4\\
\end{pmatrix}
\begin{pmatrix}
1&2
\end{pmatrix} = 
\begin{pmatrix}
1 &2\\
2&4\\
3&6\\
4&8\\
\end{pmatrix}
\end{equation*}
Итого:

\begin{equation*}
\begin{pmatrix}
3 &6\\
12&9\\
3&3\\
3&-3\\
\end{pmatrix}+
\begin{pmatrix}
  1&0\\
1&2\\
2&1\\
1&3\\
\end{pmatrix}-
\begin{pmatrix}
1 &2\\
2&4\\
3&6\\
4&8\\
\end{pmatrix}=
\end{equation*}


\begin{equation*}=
\begin{pmatrix}
3+1-1 &6+0-2\\
12+1-2&9+2-4\\
3+2-3&3+1-6\\
3+1-4&-3+3-8\\
\end{pmatrix}=
\begin{pmatrix}
3 &4\\
11&7\\
2&-2\\
0&-8\\
\end{pmatrix}
\end{equation*}
\subsection*{Ответ:}
\begin{equation*}
\begin{pmatrix}
3 &4\\
11&7\\
2&-2\\
0&-8\\
\end{pmatrix}
\end{equation*}
















\section*{2}
Вычислим максимально рациональным способом:\\
 tr((A + B)(X + Y) - XA - XB - YB)  =\\
= tr( AX + AY + BX + BY - XA - XB - YB) = \\
= tr(AX) + tr(AY) + tr(BX) + tr(BY) - tr(XA) - tr(XB) - tr(YB) = \\
= tr(AX) + tr(AY) + tr(BX) + tr(BY) - tr(AX) - tr(BX) - tr(BY) = \\
= tr(AY)  
\begin{equation*}
A = 
\begin{pmatrix}
1 &2&1\\
0&1&0\\
-1&0&1\\
\end{pmatrix}    Y = 
\begin{pmatrix}
1 &2&3\\
4&1&1\\
3&2&1\\
\end{pmatrix}
\end{equation*}

\begin{equation*}
AY = 
\begin{pmatrix}
1 &2&1\\
0&1&0\\
-1&0&1\\
\end{pmatrix}  
\begin{pmatrix}
1 &2&3\\
4&1&1\\
3&2&1\\
\end{pmatrix} = 
\begin{pmatrix}
12 &6&6\\
4&1&1\\
2&0&-2\\
\end{pmatrix}   
\end{equation*}

\begin{equation*}
tr(AY) = tr
\begin{pmatrix}
12 &6&6\\
4&1&1\\
2&0&-2\\
\end{pmatrix}   = 12 +1 -2 = 11
\end{equation*}
\subsection*{Ответ: 11}





\newpage
\newpage
\newpage
\newpage






 



Составим матрицу СЛУ:
\begin{equation*}
    \left(\begin{array}{rrrr|r}
   2 & -1 & 3 & -7 &5\\
   6 & -3 & 1 & -4 & 7\\
  4 & -2 & 14 & -31 & 18 
   \end{array}\right)_{II -= I * 3}
   \approx
   \left(\begin{array}{rrrr|r}
   2 & -1 & 3 & -7 &5\\
   0 & 0 & -8 & 17 & -8\\
  4 & -2 & 14 & -31 & 18 
   \end{array}\right)_{III -= I * 2}
\approx
\end{equation*}
\begin{equation*}
\approx
   \left(\begin{array}{rrrr|r}
   2 & -1 & 3 & -7 &5\\
   0 & 0 & -8 & 17 & -8\\
   0 & 0 & 8 & -17 & 8\\
   \end{array}\right)_{I = I * \frac{3}{8}}
\approx
   \left(\begin{array}{rrrr|r}
    2 & -1 & 0 & -\frac{5}{8} &2\\
   0 & 0 & -8 & 17 & -8\\
   \end{array}\right)
  \approx
\end{equation*}
\begin{equation*}
\approx
   \left(\begin{array}{rrrr|r}
   1 & \frac{-1}{2} & 0 & -\frac{5}{16} &1\\
   0 & 0 & 1 & \frac{-17}{8} & 1\\
   \end{array}\right)
\end{equation*}

Ответ:

\begin{equation*}
\begin{pmatrix}
  x_1 \\
  x_2\\
  x_3\\
  x_4
\end{pmatrix} = 
\begin{pmatrix}
  \frac{1}{2} x_2 + \frac{5}{16}x_4 + 1\\
  x_2\\
  \frac{17}{8}x_4 + 1\\
	x_4
\end{pmatrix}
\end{equation*}

\section*{2}

Составим матрицу СЛУ и вычтем из 2,3 и 4 строчек певую
\begin{equation*}
   \left(\begin{array}{rrrr|r}
   2 & 3 & 1 & 2 & 3\\
   4 & 6 & 3 & 4 & 5\\
   6 & 9 & 5 & 6 & 7\\ 
   8 & 12 & 7 & \lambda & 9\\ 
   \end{array}\right)
\approx
   \left(\begin{array}{rrrr|r}
   2 & 3 & 1 & 2 & 3\\ 
   0 & 0 & 1 & 0 & -1\\
   0 & 0 & 2 & 0 & -2\\ 
   0 & 0 & 3 & \lambda - 8 & -3\\ 
   \end{array}\right)_{IV -= 3II}
  \approx
\end{equation*}



\begin{equation*}
\approx
   \left(\begin{array}{rrrr|r}
   1 & \frac{3}{2} & 0 & 1 & 2\\ 
   0 & 0 & 1 & 0 & -1\\
   0 & 0 & 0 & \lambda - 8 & 0\\ 
   \end{array}\right)
\end{equation*}

При $\lambda = 8$ $x_4 $ - свободная переменная.
Тогда ответ принимает вид:

\begin{equation*}
\begin{pmatrix}
  x_1 \\
  x_2\\
  x_3\\
  x_4\\
  \lambda
\end{pmatrix} = 
\begin{pmatrix}
  -\frac{3}{2}x_2 - x_4 + 2\\
x_2\\
	-1\\
	x_4\\
	8
\end{pmatrix}
\end{equation*}

При $\lambda \neq 8$ $x_4 = 0$

Ответ:

\begin{equation*}
\begin{pmatrix}
  x_1 \\
  x_2\\
  x_3\\
  x_4\\
\end{pmatrix} = 
\begin{pmatrix}
  -\frac{3}{2}x_2 + 2\\
x_2\\
	-1\\
	0\\
\end{pmatrix}
\end{equation*}

\section*{3}


\begin{equation*}
 \begin{cases}
  a0^3 + b0^2 + c0 + d = 2\\
  a1^3 + b1^2 + c1 + d = 3\\
  a2^3 + b2^2 + c2 + d = 1\\
  a3^3 + b3^2 + c3 + d = -7\\
 \end{cases}
\end{equation*}


\begin{equation*}
   \left(\begin{array}{rrrr|r}
   0 & 0 & 0 & 1 & 2\\ 
   1 & 1 & 1 & 1 & 3\\
   8 & 4 & 2 & 1 & 1\\ 
   27 & 9 & 3 & 1 & -7\\ 
   \end{array}\right)
\approx
  \left(\begin{array}{rrrr|r}
   1 & 1 & 1 & 0 & 1\\
   8 & 4 & 2 & 0 & -1\\ 
   27 & 9 & 3 &0 & -9\\ 
    0 & 0 & 0 & 1 & 2\\ 
   \end{array}\right)
  \approx
\end{equation*}

\begin{equation*}
\approx
 \left(\begin{array}{rrrr|r}
   1 & 1 & 1 & 0 & 1\\
   0 & -4 & -6 & 0 & -9\\ 
   0 & -18 & -24 & 0 & -36\\ 
    0 & 0 & 0 & 1 & 2\\ 
   \end{array}\right)
\approx
 \left(\begin{array}{rrrr|r}
   1 & 1 & 1 & 0 & 1\\
   0 & 4 & 6 & 0 & 9\\ 
   0 & 3 & 4 & 0 & 6\\ 
    0 & 0 & 0 & 1 & 2\\ 
   \end{array}\right)_{II -= III}
  \approx
\end{equation*}

\begin{equation*}
\approx
 \left(\begin{array}{rrrr|r}
   1 & 1 & 1 & 0 & 1\\
   0 & 1 & 2 & 0 & 3\\ 
   0 & 3 & 4 & 0 & 6\\ 
    0 & 0 & 0 & 1 & 2\\ 
   \end{array}\right)_{III -= 3II}
\approx
 \left(\begin{array}{rrrr|r}
   1 & 1 & 1 & 0 & 1\\
   0 & 1 & 2 & 0 & 3\\ 
   0 & 0 & -2 & 0 & -3\\ 
    0 & 0 & 0 & 1 & 2\\
   \end{array}\right)
  \approx
\end{equation*}

\begin{equation*}
\approx
 \left(\begin{array}{rrrr|r}
   1 & 1 & 1 & 0 & 1\\
   0 & 1 & 0 & 0 & 0\\ 
   0 & 0 & 1 & 0 & \frac{3}{2}\\ 
    0 & 0 & 0 & 1 & 2\\
   \end{array}\right)
\approx
 \left(\begin{array}{rrrr|r}
   1 & 0 & 0 & 0 & -\frac{1}{2}\\
   0 & 1 & 0 & 0 & 0\\ 
   0 & 0 & 1 & 0 & \frac{3}{2}\\ 
    0 & 0 & 0 & 1 & 2\\
   \end{array}\right)
  \approx
\end{equation*}
Искомое уравнение: $-\frac{1}{2}x^3 + \frac{3}{2}x + 2$

\section*{4}

Найдем точку P:
\begin{equation*}
 \begin{cases}
  x + 2y = 5\\
  -2x + 3y = 4\\
 \end{cases}
\end{equation*}

\begin{equation*}
   \left(\begin{array}{rr|r}
   1 & 2& 5\\ 
   -2 & 3 & 4\\
   \end{array}\right)
\approx
   \left(\begin{array}{rr|r}
   1 & 2& 5\\ 
   0 & 7 & 14\\
   \end{array}\right)
  \approx
     \left(\begin{array}{rr|r}
   1 & 2& 5\\ 
   0 & 1 & 2\\
   \end{array}\right)
  \approx
       \left(\begin{array}{rr|r}
   1 & 0& 1\\ 
   0 & 1 & 2\\
   \end{array}\right)
\end{equation*}
\begin{equation*}
     \left(\begin{array}{rr|r}
   x_1\\
   y_1
   \end{array}\right)
  =
       \left(\begin{array}{rr|r}
    1\\ 
    2\\
   \end{array}\right)
\end{equation*}

Найдем точку Q:
\begin{equation*}
 \begin{cases}
  2x + -y = 4\\
  3x - 2y = 6\\
 \end{cases}
\end{equation*}

\begin{equation*}
   \left(\begin{array}{rr|r}
   2 & -1 & 4\\ 
   3 & -2 & 6\\
   \end{array}\right)
\approx
   \left(\begin{array}{rr|r}
	2 & -1 & 4\\ 
   1 & -1 & 2\\
   \end{array}\right)
  \approx
     \left(\begin{array}{rr|r}
	0 & 1 & 0\\ 
   1 & -1 & 2\\
   \end{array}\right)
  \approx
       \left(\begin{array}{rr|r}
   1 & 0& 2\\ 
   0 & 1 & 0\\
   \end{array}\right)
\end{equation*}
\begin{equation*}
     \left(\begin{array}{rr|r}
   x_2\\
   y_2
   \end{array}\right)
  =
       \left(\begin{array}{rr|r}
    2\\ 
    0\\
   \end{array}\right)
\end{equation*}

Найдем коэффициенты прямой $x = ay + b$.
\begin{equation*}
\begin{cases}
2a + b = 1
0a + b = 2
\end{cases}
\end{equation*}
Вычтем второе равенство из первого:
\begin{equation*}
\begin{cases}
2a = -1
b = 2
\end{cases}
\end{equation*}
Следовательно, искомое уравнение имеет вид $-\frac{1}{2}y + 2 = x$


\section*{5.A}
Приведем пример неопределенной системы, у которой есть только одно целое решение: Предположим, что $x_2$ - свободная переменная в этой системе, а $x_1 = 1 + (x_2-1)\sqrt{2}$ 
\begin{equation*}
     \left(\begin{array}{r}
   x_1\\
   x_2
   \end{array}\right)
  =
       \left(\begin{array}{r}
    1 + (x_2-1)\sqrt{2}\\ 
    x_2\\
   \end{array}\right)
\end{equation*}

Так как при умножении любого целого числа на $\sqrt{2}$ получается иррациональное число, $x_1$ будет целым только при $x_2 = 1$. Поэтому у этого уравнения есть только одно целое решение, в котором $x_1 = 0, x_2 = 1$.

\section*{5.Б}
Если в СЛУ используются только целые коэффициенты, в ответе могут появиться рациональные числа (в результате применения III типа преобразования).
Напишем пример такой системы:
\begin{equation*}
     \left(\begin{array}{r}
   x_1\\
   x_2
   \end{array}\right)
  =
       \left(\begin{array}{r}
    \frac{p_0}{q_0} + \frac{p}{q}  x_2\\ 
    x_2\\
   \end{array}\right)
 \approx
 \left(\begin{array}{r}
    \frac{qp_0}{qq_0} + \frac{q_0px_2}{qq_0} \\ 
    x_2\\
   \end{array}\right)
   \approx
 \left(\begin{array}{r}
    \frac{qp_0 + q_0px_2}{qq_0}\\ 
    x_2\\
   \end{array}\right)
\end{equation*}
Допустим, это уравнение имеет только одно решение в целых числах при $x_2 = a$. Но тогда у этого уравнения будет еще одно целое решение при $x_2 = a + nqq_0$, где n - любое целое число:
$$\frac{qp_0 + q_0p(a + nqq_0)}{qq_0} ==\frac{qp_0 + q_0pa}{qq_0} + nq_0p $$
Если в уравнении больше одной свободной переменной, можно повторить подобные размышления для каждой переменной и приравнять все свободные переменные одному числу, равному произведению $n_1q_1q_{01} * n_2q_2q_{02} * ... * n_mq_mq_{0m}$
Следовательно, у уравнения не может быть одно и только одно решение в целых числах.

\section*{6}
Рассмотрим используемые преобразования в алгоритме  Гаусса для приведения к ступенчатому виду:


Преобразования I не влияют на другие преобразования, поэтому их можно свободно перемещать: например, заранее расставить строки в нужном порядке до использования преобразований II и III типа.


Допустим, в процессе решения мы в начале применили преобразование III вида (добавив к строчке i строчку j, домноженную на n), а затем применили преобразование II вида(умножив строчку j на m).
Но мы можем поменять местами эти преобразования, если а начале домножим строчку j на m, а затем добавив к строчке i строчку j, домноженную на $\frac{n}{m}$.

Если на m умножили вначале строчку i, то при преобразовании III вида нужно умножать строчку на m*n.
Эти перестановки не влияют на алгоритм Гаусса: все элементы в столбце кроме ведущего будут обнулены.


Таким образом, любую цепочку элементарных преобразований можно преобразить в вид n преобразований I -> m преобразований II -> c преобразований III.



\end{document}