\documentclass[a4paper]{article}


\usepackage[T2A]{fontenc}
\usepackage[utf8]{inputenc}
\usepackage[english,russian]{babel}


\usepackage{amsmath, amsfonts, amssymb, amsthm, mathtools}

\author{Шорин Сергей, БКНАД211}
\title{Линейная Алгебра ДЗ 4}
\date{\today}

\begin{document}

\maketitle

\newpage

\section*{1.a}
Переведите из алгебраического вида в тригонометрический: $ -\sqrt{3} + i$

Вынесем число 2($ \sqrt{\sqrt{3}^2 + 1^2} = 2$)

$$2 ( -\frac{\sqrt{3}}{2} + \frac{1}{2}i)$$

Нам подходит угол $\varphi = \frac{5}{6} \pi$

$$ -\sqrt{3} + i = 2(\cos(\frac{5}{6} \pi) + i\sin(\frac{5}{6} \pi)$$



\section*{1.б}

\section*{1.в}

\section*{2.a}
Вычислить $\frac{(1 + 3i)(8 - i)}{(2 + i)^ 2}$

$$\frac{(1 + 3i)(8 - i)}{(2 + i)^ 2}= \frac{11 + 23i}{3 + 4i} = \frac{(11 + 23i)(3 - 4i)}{(3 + 4i)(3 - 4i)}= \frac{33 -44i + 69i + 92}{25} = \frac{125 -44i}{25}$$

\section*{2.в}

\section*{3.a}

\section*{3.б}

\section*{3.в}
\section*{4.a}
Вычислите, применив формулу Муавра: $(1 + i \sqrt{3})^{150}$

Переведем в тригонометрический вид:

$$(1 + i \sqrt{3})^{150} = 2^{150}(\cos(\frac{\pi}{3}) + i\sin(\frac{\pi}{3}))^{150} = 2^{150}(\cos(50\pi) + i\sin(50\pi)) = 2^{150}$$


\section*{4.б}


\end{document}