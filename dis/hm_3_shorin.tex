\documentclass[a4paper]{article}


\usepackage[T2A]{fontenc}
\usepackage[utf8]{inputenc}
\usepackage[english,russian]{babel}


\usepackage{amsmath, amsfonts, amssymb, amsthm, mathtools}

\author{Шорин Сергей, БКНАД211}
\title{Дискретная математика дз 3}
\date{\today}

\begin{document}

\maketitle

\newpage

\section*{1.a}
Сколько существует чётных шестизначных натуральных чисел, в десятичной записи которых есть хотя бы одна цифра 7?


Рассмотрим, сколько всего есть шестизначных четных чисел. На первом месте могут стоять числа 1..9, в середине 0..9, в конце - 0, 2, 4, 6, 8.

$$9 * 10 * 10 * 10 * 10 * 5 = 450000$$

Рассмотрим, сколько существует четных шестизначных чисел, в записи которых нет цифры 7:

$$8 * 9 * 9 * 9 * 9 * 5 = 262440$$

Следовательно, всего есть $450000 - 262440 = 187560 $ четных шестизначных чисел, в записи которых есть хоть одна цифра 7.
\subsection*{Ответ:}
Таких чисел всего 187560.






\section*{1.b}
Сколько существует чётных шестизначных натуральных чисел, в десятичной записи которых есть хотя бы одна цифра 8?


Рассмотрим, сколько всего есть шестизначных четных чисел. На первом месте могут стоять числа 1..9, в середине 0..9, в конце - 0, 2, 4, 6, 8.

$$9 * 10 * 10 * 10 * 10 * 5 = 450000$$

Рассмотрим, сколько существует четных шестизначных чисел, в записи которых нет цифры 8:

$$8 * 9 * 9 * 9 * 9 * 4 = 209952$$

Следовательно, всего есть $450000 - 209952 = 240048 $ четных шестизначных чисел, в записи которых есть хотя бы одна цифра 8.
\subsection*{Ответ:}
Таких чисел всего 240048.

\section*{2}
Покажите, что для любых множеств $A_1, A_2, B_1, B_2$ выполнено:

$$ (A_1 \setminus A_2) \star (B_1 \setminus B_2) \subseteq (A_1 \star B_1) \setminus (A_2 \star B_2) $$

$$ (A_1 \setminus A_2) \star (B_1 \setminus B_2) = ((x \in A_1) \wedge \neg  (x \in A_2)) \wedge ((y \in B_1) \wedge \neg (y \in B_2))$$

$$ (A_1 \star B_1) \setminus (A_2 \star B_2) = ((x \in A_1) \wedge (y \in B_1)) \wedge \neg  ((x \in A_2) \wedge (y \in B_2)) = $$

$$ = ((x \in A_1) \wedge (y \in B_1)) \wedge   (\neg(x \in A_2) \vee \neg(y \in B_2)) $$
Теперь сравним множества:

$$((x \in A_1) \wedge \neg  (x \in A_2)) \wedge ((y \in B_1) \wedge \neg (y \in B_2)) \overset{?}{\subseteq } ((x \in A_1) \wedge (y \in B_1)) \wedge   (\neg(x \in A_2) \vee \neg(y \in B_2))$$

Уберем $(x \in A_1) \wedge (y \in B_1))$ из обоих частей:

$$\neg  (x \in A_2) \wedge   \neg (y \in B_2) \overset{?}{\subseteq } \neg(x \in A_2) \vee \neg(y \in B_2)$$

Так как для левого множеств требуется и отсутствие x в $A_2$, и отсутствие y в $B_2$, а правое множество требует выполнения только одного из этих условий - мощность левого множества меньше или равна правому множеству, что и требовалось доказать.


\section*{3}
Сколько существует слов длины n в алфавите {a, b, c}, в которых присутствует
каждая из букв a, b, c?

Пусть $A$  - множество всех слов, в которых есть буква а, $B$  - множество всех слов, в которых есть буква b, $C$  - множество всех слов, в которых есть буква c.

Тогда:

$$ |A \setminus B \setminus C| = 1 (aa..a)$$
$$ |B \setminus A \setminus C| = 1 (bb..b)$$
$$ |C \setminus B \setminus A| = 1 (cc..c)$$

$$ |(A \cup B) \setminus C| = 2^n$$
(Для B и C аналогично)
$$ |A \cup B \cup C| = 3_n$$
Тогда:
$$|A \cap B \cap C|  = |A \cup B \cup C| - |(A \cup B) \setminus C| - |(B \cup C) \setminus A| - |(C \cup A) \setminus B| + |A \setminus B \setminus C| +|B \setminus A \setminus C| + |C \setminus B \setminus A| $$
$$|A \cap B \cap C| = 3^n - 3* 2^n +3$$

\subsection*{Ответ}
$$3^n - 3 * 2^n + 3$$


\section*{4}

Сколько существует целых чисел от 1 до $10^6$ $ \textbf{включительно}$ , которые не являются ни полным квадратом, ни полным кубом, ни четвёртой степенью?

Найдем мощность множества чисел, для которых не выполняются заданные условия.

Воспользуемся формулой включения-исключения для трех множеств:

Пусть $A$  - множество целых чисел, которые являются полным квадратом, $B$  - множество целых чисел, которые являются полным кубом, $C$  - множество целых чисел, которые являются четвертой степенью некоторого числа.

Тогда:

$$|A \cup B \cup C| = |A| + |B| + |C| - |A \cap B| - |A \cap C| - |B \cap C| + |B \cap C \cap A| $$
$$|A| =  |{1^2, 2^2, 3^2, ..., 1000^2}| = 1000 $$
$$|B| =  |{1^3, 2^3, 3^3, ..., 100^3}| = 100 $$
$$|C| =  |{1^4, 2^4, 3^4, ..., 31^4}| = 31 $$
Если целое число является и полным квадратом, и полным кубом, то оно является шестой степенью:
$$|A \cap B| = |{1^6, ... 10^6}| = 10$$
Любое число, являющееся четвертой степенью, так же является полным квадратом:, и , то оно является четвертой степенью:
$$|A \cap C| = |{1^4, 2^4, 3^4, ..., 31^4}| = 31$$
Если целое число является и полным кубом, и четвертой степенью, то оно является двенадцатой степенью степенью:
$$|B \cap C| = |{1^{12}, 2^{12}, 3^{12}}| = 3$$

Если целое число является и полным квадратом, и полным кубом, и четвертой степенью, то оно является двенадцатой степенью степенью:
$$|A \cap B \cap C| = |{1^{12}, 2^{12}, 3^{12}}| = 3$$

$$|A \cup B \cup C| = |A| + |B| + |C| - |A \cap B| - |A \cap C| - |B \cap C| + |B \cap C \cap A|  = 1000 + 100 + 31 - 10 - 31 -3 + 3 = 1090$$

Всего чисел, удовлетворяющих условию задачи: $1000000 - 1090 = 998910$

\subsection*{Ответ}
$$998910$$




\section*{5}

Сколько существует целых чисел от 1 до 33000, не делящихся ни на 3, ни на 5,
ни на 11?

Найдем мощность множества чисел, для которых не выполняются заданные условия.

Воспользуемся формулой включения-исключения для трех множеств:

Пусть $A$  - множество целых чисел, которые не делятся на 3, $B$  - множество целых чисел, которые не делятся на 5, $C$  - множество целых чисел, не делятся на 11.

Тогда:

$$|A \cup B \cup C| = |A| + |B| + |C| - |A \cap B| - |A \cap C| - |B \cap C| + |B \cap C \cap A| $$
$$|A| =  |{3 * 1, 3 * 2, ..., 3 * 11000}| = 11000 $$
$$|B| =  |{5 * 1, 5 * 2, ..., 5 * 6600}| = 6600 $$
$$|C| =   |{11 * 1, 11 * 2, ..., 11 * 3000}| = 3000 $$
$$|A \cap B| = |{15 * 1, ... 15 * 2200}| = 2200$$
$$|A \cap C| = |{33 * 1, 33 * 2, ..., 33 * 1000}| = 1000$$
$$|B \cap C| = |{55* 1, 55 * 600}| = 600$$
$$|A \cap B \cap C| = |{165* 1, 165 * 200}| = 200$$
$$|A \cup B \cup C| = |A| + |B| + |C| - |A \cap B| - |A \cap C| - |B \cap C| + |B \cap C \cap A|  = 11000 + 6600 + 3000 - 2200 - 1000 -600 + 200 = 17000$$

Всего чисел, удовлетворяющих условию задачи: $33000 - 17000 = 16000$

\subsection*{Ответ}
$$16000$$




\end{document}