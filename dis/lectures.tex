\documentclass[a4paper]{article}


\usepackage[T2A]{fontenc}
\usepackage[utf8]{inputenc}
\usepackage[english,russian]{babel}


\usepackage{amsmath, amsfonts, amssymb, amsthm, mathtools}

\author{Шорин Сергей, БКНАД211}
\title{Матанализ ДЗ 1}
\date{\today}

\begin{document}

\maketitle

\newpage

\section{Лекция 1 Введение в алгебру логики}



Определение: Высказывание - это утверждение, о котором можно однозначно сказать оно истинное или нет. Выскзывание можно обозначить латинской буквой.

Высказывание можно объединять в связки - A и B

Высказывание без связок - простое. С связками - составное.


Таблицы истинности:

Кноьюнкция: $a\wedge b$ = a and b

Дизюнкция: $a \vee b$ = a or b

Свойства: куммутативны 

Импликация: $a \rightarrow b$:
\begin{equation}
\begin{tabular}{cc|c}
a & b & a $\rightarrow$ b \\
\hline
0 & 0 & 1 \\
0 & 1 & 1 \\
1 & 0 & 0 \\
1 & 1 & 1 \\
\end{tabular}
\end{equation}


Равносильность: $a \equiv b$:


\begin{equation}
\begin{tabular}{cc|c}
a & b & a $\equiv$ b \\
\hline
0 & 0 & 1 \\
0 & 1 & 0 \\
1 & 0 & 0 \\
1 & 1 & 1 \\
\end{tabular}
\end{equation}


Отрицание: $\neg a$:
\begin{equation}
\begin{tabular}{c|c}
a & $\neg$ a \\
\hline
0 & 1 \\
1 & 0 \\
\end{tabular}
\end{equation}



Законы Де Моргана:

$\neq a \wedge \neg b = a \vee  b $

Тавтология: высказывание, которое всегда истинное

противоречие: высказывание, которое всегда ложно



\section*{Введение в теорию множеств}

Упрощенное определение множества: Множество - совокупность объектов или элементов.

Способы задания множества:

\{Описание множества \}

\{x | x  обладает свойством p\}

$\emptyset$ - пусто множество

Определение: Если a есть один из элементов множества A, то мы говорим, что a есть элемент множества A и a принадлежит A.

$$ a \in A$$

$$ a \not\in A$$

$$ A \cap B: \{x| ( x \in A) \wedge (x \in B)\}$$


$$ A \cup B: \{x| ( x \in A) \vee (x \in B)\}$$
$$ A \\ B: \{x| ( x \in A) \wedge \neg(x \in B)\}$$




\end{document}