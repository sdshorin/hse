\documentclass[a4paper]{article}


\usepackage[T2A]{fontenc}
\usepackage[utf8]{inputenc}
\usepackage[english,russian]{babel}


\usepackage{amsmath, amsfonts, amssymb, amsthm, mathtools}

\author{Шорин Сергей, БКНАД211}
\title{Матанализ ДЗ 1}
\date{\today}

\begin{document}

\maketitle

\newpage

\section{}
a) Доказать иррациональность числа $\sqrt{2} + \sqrt{3} + \sqrt{5}$

Предположим, что это выражение равно рациональному числу $\frac{ p}{q}$. Тогда:
\[\sqrt{2} + \sqrt{3} + \sqrt{5} = \frac{p}{q}\]
\[\sqrt{2} + \sqrt{3}  = \frac{p}{q} - \sqrt{5}\]
\[(\sqrt{2} + \sqrt{3})^2  = (\frac{p}{q} - \sqrt{5})^2\]
\[ 5 +  2\sqrt{6} = \frac{p^2}{q^2} + 5 - 2\sqrt{5}\frac{p}{q}\]
\[ (2\sqrt{6} + 2\sqrt{5}\frac{p}{q})^2= \frac{p^4}{q^4}\]
\[ 24 + 20\frac{p^2}{q^2} + 8\sqrt{30}\frac{p}{q}= \frac{p^4}{q^4}\]
\[  8\sqrt{30}\frac{p}{q}= \frac{p^4}{q^4} - 24 -  20\frac{p^2}{q^2}\]

\[  8\sqrt{30}= \frac{p^3}{q^3} - 24\frac{q}{p} -  20\frac{p}{q}\]
Очевидно, что если $\frac{ p}{q}$ рациональное число, то $\frac{p^3}{q^3} - 24\frac{q}{p} -  20\frac{p}{q}$ - тоже рациональное число. Заменим $\frac{p^3}{q^3} - 24\frac{q}{p} -  20\frac{p}{q}$ на $8\frac{p_1}{q_1}$
\[  \sqrt{30}= \frac{p_1}{q_1}\]
Тогда
\[  30 = \frac{p_1^2}{q_1^2}\]
\[  30 p_1^2= q_1^2\]
Следовательно, $q_1$ делится на 30, значит, $q_1$ - четное. Заменим $q_1$ на 2k.
\[  30 p_1^2= (2k)^2\]
\[  15 p_1^2= 2k^2\]
Следовательно, $p_1$ - четное. Но так быть не может, ведь $\frac{p_1}{q_1}$ - рациональное число, то есть несократимая дробь. Следовательно, $\sqrt{2} + \sqrt{3} + \sqrt{5}$ - иррациональное число.

\newpage
б) Доказать иррациональность числа $\log_219$


Предположим, что это выражение равно рациональному числу $\frac{ p}{q}$. Тогда:
\[\log_219 = \frac{ p}{q}\]
\[2^{\frac{ p}{q}} = 19\]
\[2^p = 19^q, q \neq 0\]
Так как 19 не делится на 2, не существует таких целых p и q, при которых это равенство верно. Следовательно, $\frac{ p}{q}$ - не рациональное число. Поэтому $\log_219$ - иррациональное число.

\section{}


a) Найти сумму: $\sum^{99}_{k=1}k\cdot k!$


Преобразуем:
\[k\cdot k!\Rightarrow\]
\[(k + 1 -1)\cdot k!\Rightarrow\]
\[(k + 1)\cdot k! - k!\Rightarrow\]
\[ - k!  + (k + 1)! \]
\[\sum^{99}_{k=1}k\cdot k! = \sum^{100}_{k=2}k! - \sum^{99}_{k=1}k! = 100! -1\]


б) Найти сумму $\frac{1}{2} + \frac{3}{2^2} + \frac{5}{2^3} + ... + \frac{2n -1}{2^n}$

Обозначим искомую сумму за S:
\begin{equation}\label{aaa}
\frac{1}{2} + \frac{3}{2^2} + \frac{5}{2^3} + ... + \frac{2n -1}{2^n} = S
\end{equation}
Домножим обе части уравнения на 2:
\[1 + \frac{3}{2} + \frac{5}{2^2} +\frac{7}{2^3}+ ... + \frac{2n -1}{2^{n-1}} = 2S \]
\[1 + \frac{3}{2} + (\frac{2}{2^2} + \frac{3}{2^2}) +(\frac{2}{2^3} + \frac{5}{2^3})+ ... +(\frac{2}{2^{n-1}} + \frac{2n - 3}{2^{n-1}}) = 2S \]
\[(2 +\frac{1}{2}) + (\frac{1}{2} + \frac{3}{2^2}) +(\frac{1}{2^2} + \frac{5}{2^3})+ ... +(\frac{1}{2^{n-2}} + \frac{2n - 3}{2^{n-1}}) = 2S \]
Вычтем равенство \eqref{aaa}:
\[2 + (\frac{1}{2}) +(\frac{1}{2^2})+ ... +(\frac{1}{2^{n-2}})+ \frac{1}{2^{n-1}} + \frac{1}{2^n} = S \]
Применим функцию суммы геометрической прогрессии:
\[2 + (\frac{1}{2}) +(\frac{1}{2^2})+ ... +(\frac{1}{2^{n-2}}) = S \]

\[2 + (\frac{\frac{1}{2}(\frac{1}{2^n} - 1)}{-\frac{1}{2}})= S \]
\[2 - (\frac{1}{2^n} - 1)= S = 3 \]
\newpage

\section{}
a) Доказать, используя индукцию: $1^3 + 2^3 + ... + n^3 = (\frac{n(n + 1)}{2})^2$\\
В качестве базы индукции возьмем число 2:
\[1 + 8 = (\frac{2(2 + 1)}{2})^2 = 3^2 = 9\]
Добавим к обоим частям уравнения $(n + 1)^3$:
\[1^3 + 2^3 + ... + n^3 + (n + 1)^3 = (\frac{n(n + 1)}{2})^2 + (n + 1)^3\]

\[1^3 + 2^3 + ... + n^3 + (n + 1)^3 = \frac{n^2(n + 1)^2}{4} + \frac{4(n + 1)^3}{4}\]

\[1^3 + 2^3 + ... + n^3 + (n + 1)^3 = \frac{(n + 1)^2(n^2 + 4n + 4)}{4}\]

\[1^3 + 2^3 + ... + n^3 + (n + 1)^3 = \frac{(n + 1)^2(n+ 2)^2}{4}\]

\[1^3 + 2^3 + ... + n^3 + (n + 1)^3 = (\frac{(n + 1)(n+ 2)}{2})^2\]
Что и требовалось доказать.

б) Доказать, используя индукцию: $ \sqrt{n} < \sum_{k=1}^n\frac{1}{\sqrt{k}} < 2 \sqrt{n}, n \geqslant 2 $

База индукции, n = 2:
$$ \sqrt{2} < 1 + \frac{1}{\sqrt{2}} < 2 \sqrt{2}$$
$$ 2 < \sqrt{2} + 1 < 4 $$
$$ 2 < 2.4142... < 4 $$

При $n = 2$ неравенство верно. Теперь докажем, что если при $n$ неравенство верно, то и при $n+ 1$ неравенство верно.

\begin{equation}\label{bbb}
\sqrt{n + 1} \overset{?}{<} \sum_{k=1}^n\frac{1}{\sqrt{k}} + \frac{1}{\sqrt{n+1}} \overset{?}{<} 2 \sqrt{n+1}
\end{equation}
Вычтем исходное неравенство. Так как из второго элемента неравенства мы вычтем большее значение, чем из первого(а из третьего - еще большее), то в дальнейшем мы будем доказывать более сильное утверждение. Если в новом неравенстве отношения будут $... < ... < ...$, то следовательно, в неравенстве \eqref{bbb} они будут такие же.

\begin{equation}
\sqrt{n + 1} - \sqrt{n } \overset{?}{<} + \frac{1}{\sqrt{n+1}} \overset{?}{<} 2 \sqrt{n+1} - \sqrt{n }
\end{equation}

Домножим на $\sqrt{n+1}$:

\begin{equation}
n + 1 - \sqrt{n^2 + n } \overset{?}{<} 1 \overset{?}{<} 2n + 2 - 2\sqrt{n^2 + n }
\end{equation}


Рассмотрим неравенство $n + 1 - \sqrt{n^2 + n } \overset{?}{<} 1$:
$$n  \overset{?}{<} \sqrt{n^2 + n }$$
Так как обе части больше 0, возведем в квадрат:
$$n^2  \overset{?}{<} n^2 + n $$
$$0  < n $$
что и требовалось доказать.

Теперь рассмотрим второе неравенство $ 1 \overset{?}{<} 2n + 2 - 2\sqrt{n^2 + n }$:

$$2\sqrt{n^2 + n }   \overset{?}{<} 2n + 1  $$
$$4n^2 + 4n    \overset{?}{<} 4n^2 + 4n + 1  $$
$$0   < 1  $$
Что и требовалось доказать.
\end{document}
