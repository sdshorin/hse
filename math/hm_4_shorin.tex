\documentclass[a4paper]{article}


\usepackage[T2A]{fontenc}
\usepackage[utf8]{inputenc}
\usepackage[english,russian]{babel}


\usepackage{amsmath, amsfonts, amssymb, amsthm, mathtools}

\author{Шорин Сергей, БКНАД211}
\title{Матанализ ДЗ 4}
\date{\today}

\begin{document}

\maketitle



\section*{1.a}
Исследовать на сходимость с помощью критерия Коши $\sum^{\infty}_{n=1}  \frac{\cos (2 ^ n)}{n^2}$

$$a_i = \sum^{i}_{n=1}  \frac{\cos (2 ^ n)}{n^2}$$

$$ |a_{n + p} - a_n |  < \epsilon$$

$$ \frac{\cos 2 ^ {(n + 1)}}{(n + 1)^2}
+ \frac{\cos 2 ^ {(n + 2)}}{(n + 2)^2}
...
+ \frac{\cos 2 ^ {(n + p)}}{(n + p)^2} <  \epsilon$$

$$ \frac{\cos 2 ^ {(n + 1)}}{(n + 1)^2}
+ \frac{\cos 2 ^ {(n + 2)}}{(n + 2)^2}
...
+ \frac{\cos 2 ^ {(n + p)}}{(n + p)^2}
<
\frac{1}{(n + 1)^2}
+ \frac{1}{(n + 2)^2}
...
+ \frac{1}{(n + p)^2} < 
\sum^{i}_{n=1}  \frac{1}{n^2}
<
\epsilon$$
Так как ряд $\sum^{i}_{n=1}  \frac{1}{n^2}$ сходится, исходный ряд так же сходится.

\section*{2.a}
Исследовать ряд на сходимость и найти сумму в случае сходимости: $\sum^{\infty}_{n=1}  \frac{1}{n(n + 1)(n + 2)}$

Так как каждый элемент ряда $a_n$ строго меньше элемента   $b_n$ сходящейся последовательности $ B = \sum^{\infty}_{n=1}  \frac{1}{n^2}$ (и больше 0), ряд сходится. Найдем его сумму:

$$\frac{1}{n(n + 1)(n + 2)} = \frac{1 + (n + 1) - (n+ 1)}{n(n + 1)(n + 2)} = \frac{n + 2}{n(n + 1)(n + 2)} - \frac{ n + 1}{n(n + 1)(n + 2)} = $$

$$ = \frac{1}{n(n + 1)} - \frac{ 1}{n(n + 2)} = 
\frac{1 + n - n}{n(n + 1)} - \frac{ 1 + n - n}{n(n + 2)}
=
\frac{1 + n}{n(n + 1)} - \frac{n}{n(n + 1)} - \frac{ 1 + n}{n(n + 2)} + \frac{  n}{n(n + 2)}
$$

$$= \frac{1}{n} - \frac{1}{(n + 1)} - \frac{ 1 + n}{n(n + 2)} + \frac{  1}{(n + 2)}=
\frac{1}{n} - \frac{1}{(n + 1)} - \frac{ 1}{n(n + 2)} - \frac{ 1}{(n + 2)}+ \frac{  1}{(n + 2)} = 
$$

Разделим последовательность на 2 и рассмотрим 1 подпоследовательность:
$$= \frac{1}{n} - \frac{1}{(n + 1)} - \frac{ 1}{n(n + 2)} = (\frac{1}{n} - \frac{1}{(n + 1)}) - \frac{1}{2}(\frac{1}{n} - \frac{1}{(n + 2)})  
$$
При $n \in [1, \infty]$
$$ \frac{1}{1} - \frac{1}{(1 + 1)} + \frac{1}{1 + 1} - \frac{1}{(1 + 2)} + ... + \frac{1}{(n \to \infty)} = 1$$

Рассмотрим 2 подпоследовательность:
$$\frac{1}{n} - \frac{1}{(n + 2)} + \frac{1}{n + 1} - \frac{1}{(n + 3)} + \frac{1}{n + 2} - \frac{1}{(n + 4 )} +  \frac{1}{n + 3} - \frac{1}{(n + 5 )} + ... + \frac{1}{(n \to \infty)} = \frac{1}{n} + \frac{1}{n + 1} + \frac{1}{(n \to \infty)}$$

При $n \in [1, \infty]$
$$\frac{1}{1} + \frac{1}{2}$$


$$\sum^{\infty}_{n=1}  \frac{1}{n(n + 1)(n + 2)} = 1 - \frac{1}{2}(1 + \frac{1}{2}) = 1 - \frac{3}{4} = \frac{1}{4}$$

\subsection*{Ответ:}
$$\sum^{\infty}_{n=1}  \frac{1}{n(n + 1)(n + 2)} = \frac{1}{4}$$




\section*{2.b}
Исследовать ряды на сходимость и найти суммы в случае сходимости: $\sum^{\infty}_{n=1}  \frac{1}{\sqrt{n} + \sqrt{n + 1} + \sqrt{n + 2}}$

$$\sum^{\infty}_{n=1}  \frac{1}{\sqrt{n} + \sqrt{n + 1} + \sqrt{n + 2}} > \frac{1}{3} \sum^{\infty}_{n=1}  \frac{1}{\sqrt{n + 2}} =  \sum^{\infty}_{n=3}  \frac{1}{\sqrt{n}}$$
Так как ряд $\frac{1}{\sqrt{n}}$ расходится, исходный ряд так же расходится.


\subsection*{Ответ:}
Данный ряд расходится.

\section*{2.b}
Исследовать ряды на сходимость и найти суммы в случае сходимости: $\sum^{\infty}_{n=1}  \sin (nx)$

Рассмотрим случай x = 0: в таком случае ряд сходится, так как каждый его элемент равен 0, и сумма ряда равна 0: $\sin (0) + \sin (0) +\sin (0) +\sin (0)+... = 0 + 0+ 0 + 0 +... = 0$

Рассмотрим случай $ x \neq 0$:

Тогда для любого $N(\epsilon)$ найдется такое p, зависящее от N, что $|a_n - a_{n + p}| > \epsilon $.

Допустим, $p_1 = \frac{\pi}{x}, p_2 = \frac{\pi}{2x}$.
Тогда для $p_1$:

$$ \sin ((n + \frac{\pi}{x})x) = 
 \sin (nx + \pi)
 $$
$| \sin (nx ) - \sin (nx + \pi)|$ меньше $\epsilon$ только в случае, когда остаток от деления на  $2\pi$ приближается к $ \pi$ или к $ 0$,  но в таком случае $|\sin ((n + \frac{\pi}{2x})x) - \sin (nx)| > \epsilon $ при достаточно малых $\epsilon$.

Следовательно, данный ряд расходится.

\subsection*{Ответ:}
Ряд расходится.
 
\end{document}
