\documentclass[a4paper]{article}


\usepackage[T2A]{fontenc}
\usepackage[utf8]{inputenc}
\usepackage[english,russian]{babel}


\usepackage{amsmath, amsfonts, amssymb, amsthm, mathtools}

\author{Шорин Сергей, БКНАД211}
\title{Матанализ ДЗ 2}
\date{\today}

\begin{document}

\maketitle

\newpage

\section{4.a}
Найти предел, используя определение: 
$\displaystyle{\lim_{x \to \infty}} \frac{n^2 + 3n - 1}{3n^2 - 2n + 2}$

$$|\frac{n^2 + 3n - 1}{3n^2 - 2n + 2} - \frac{1}{3}| < \epsilon $$

$$\frac{3n^2 + 9n - 3 - 3n^2 + 2n - 2 }{9n^2 - 6n + 6} < \epsilon $$
$$\frac{ 11n - n}{9n^2 + n^2} < \frac{ 11n - 5}{9n^2 - 6n + 6} < \epsilon $$

$$\frac{ 10n}{10n^2} < \epsilon $$
$$\frac{ 1}{n} < \epsilon $$
$$N = \frac{ 1}{\epsilon} + 1 $$

\section{4.б}
Найти предел, используя определение: 
$\displaystyle{\lim_{x \to \infty}} \frac{\log_an}{n}$

$$ \frac{\log_an}{n} < \frac{\log_an}{n} < \epsilon $$
$$ \log_an < \epsilon n $$

$$ \frac{\ln n}{\ln a} < \epsilon n $$

$$ 1 < \ln n < \epsilon n \ln a $$
$$ 1 < \epsilon n \ln a $$
$$ \frac{1}{\epsilon \ln a} <  n  $$
$$ \frac{1}{\epsilon \ln a} <  n  $$
$$ N = \frac{1}{\epsilon \ln a} + 1  $$

\section{4.в}
Найти предел, используя определение: 
$\displaystyle{\lim_{x \to \infty}} \sqrt[n]{n}$

$$\sqrt[n]{n} - 1 < \epsilon$$

$$\sqrt[n]{n}  < \epsilon + 1$$ 

$$ n  < (1 + \epsilon ) ^ n$$ 
Заметим, что правую часть неравенства можно разложить по биному Ньютона.

$$ n  < 1 + n\epsilon + \frac{n(n-1)}{2}\epsilon^2 + ... $$ 

Так как $\epsilon$ стремится к нулю, $\epsilon^2$ 



\end{document}
