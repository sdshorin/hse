\documentclass[a4paper]{article}


\usepackage[T2A]{fontenc}
\usepackage[utf8]{inputenc}
\usepackage[english,russian]{babel}


\usepackage{amsmath, amsfonts, amssymb, amsthm, mathtools}

\author{Sergey}
\title{Матанализ}
\date{\today}

\begin{document}

\maketitle

\newpage

Оценка:\\

\title{вещественные числа}

Считаем что все свойства чисел из N и множества Q известно со школы

опр1 $a\in Q = m/n, m\in Z, $


Из обыкновенного прямоугольново треугольника с катетами 1 слежует, что что рациональных чисел недостаточно
Пусть
$x^2 = 2$
$x = p/q \in Q$ - решение
$p^2/q^2 -> p^2 = 2q^2 ->p^2 - четное ->p=2k$


Определение : Будем говорить, что множество чисел B правее мноэества чисел A, если $a <= b  для любого A из A$





Сечение Дедекинда
(существует два множества рациональных чисел, прилегающих друг другу в плотную, и между которыми нет рациональных чисел)

$Пусть A = {a: a>0, a^2 <= 2}, B = {b: b>0, b^2 >= 2}
А левее B, Так как 0<
$


опр3
Будем говорить что для множества чисел выполняется принцип полноты, если для любых его произвольных не пустых подмножеств A и B таких что A левее B, найдется разделюящи	 их элемент

для рациональных чисел принцип полноты не выполняется



(школьный материал)\\

Любое рациональное число может быть представлено периодической десятичной дробью


(не будем рассматривать десятичные дроби с периодом 9)
0.(9) = x
9.(9) = 10x
9 = 9x



Множесто иррациональных(дейсвительных) чисел:
бесконечные не переодические десятичные дроби

Множество вещественных чисел отождествялется со всеми бесконечными десятичными дробями (в том числе и переодические) вида$ +- a_0, a_1, a_2, где a_0 \in N_o, а \in {0, 1, ..., 9}, 9$ в периоде запрещено.
0.0000(0) - ноль множества действительных чисел, совпадающих с нулем

R - обозначение этого множества(иррациональных чисел)

Не нулевое число положительно, если в такой его записи стоит знак +, и отрицательным, если в его записи стоит знак -

На множестве R определены операции сложения и умножения, причем выполняются все естественные свойства этих операций -> множество вещественных чисел является полем

Вещественные числа можно сравинивать (на множестве вещественных чисел введено отнощение порядка)
Для положительных чисел: $а_0, а_1, а_2 <=  b_0, b_1, b_2 когда или a_0, a_1, a_2 == b_0, b_1, b_2, или найдется такой разряд k, что a_0 = b_0, a_1 = b_1, ..., a_k < b_k$


Для любого $a \in R  |a| = a, если a>=0,$ 


Для любых действительных чисел выполнено неравенство треугольника: $|a + b| <= |a| + |b|.$

Используя равенство $|x| = max{x, -x}, можно доказать ||a| - |b|| <=|a+b|$
Упражнение: вывести из неравенства треугольника 




Теорема 1: на множестве вещественных чисел выполнен принцип полноты.

Пусть A  и B - не пустые множества чисел, причем A левее B. Если A состоит из всех чисел a, таких что а <= 0, а B состоит из чисел >= 0, то 0 разделяет A и B.
Пусть существует a принадлеэащее A, такое что ...


\section*{Лекция 2. Пределы}

Сформулировав принцип полноты, мы можем сформулировать сложение, умножене и другие операции для действительных чисел.

Пусть $a = a_1 a_2 a_2a_4, b=b_1b_2b_3b_4$
$a+b = ?$
Определеим множество H${a_0 + b_0, a_{01} + b{01}, a_{012} + b_{012}}$

А множество B${a_0 + 1 + b_0, a_{01} + 0.1+ b{01}, a_{012} + 0.01 + b_{012}}$

A левее B -> по принципу полноты сучществует С, разделяющее A и B.
Докажем от противного:  ...
По определению: C = A + B

\section*{Предел последовательности}

Определение: Последовательность - это функция натурального аргумента. Если любому n из множества натуральных чисел поставлено соответствие некоторое число $a_n$, то говорят что задана числовая последовательность ${a_n}_{n=1}^\varpropto$.
\\
\\
Последовтельность всегда бесконечна, но она вся может быть задана одним числом.
Последовательность можно задать через формулу, рекурентно и т.д.
\\
\\
Предел последовательности рациональных может быть иррациональным.
Последовательости рациональных чисел приближают иррациональное число.
\\
\\
Определения 2: Окресностью числа a называется любой интервал, содержащий это число\\
Определение:
Эпсилон- окресностью при $\epsilon > 0$ на зывается интервал $a = \epsilon, a + \epsilon]$\\
Определение: проколотой $\epsilon$ окрестностью числа a называется $[\epsilon, a) + (a, a + \epsilon]$

Определение 3: Число A большое назаыается пределом последовательности ${a_n}_{n=1}^\varpropto$, если для любой окресности точки A, что при всех n >
N $a_n$ лежит в этой окрестности.  
\\
Дадим экивалентное определение: Говорят, что последовательность ${a_n}_{n=1}^\varpropto$ сходится к числу A, если для всякого числа $\epsilon >0$ существует такое натуральное число N, зависящее от $\epsilon$, что для любого n > N($\epsilon$) | A - $a_n$ | < $\epsilon$.\\
$A - \epsilon < a_n < A + \epsilon$
\\
\\
Пример 1:
$$a_n = \frac{1}{n} \lim a_n = 0$$
$|\frac{1}{n} - 0| \rightarrow $ Если $N(\epsilon) = [\frac{1}{\epsilon} + 1 > \frac{1}{\epsilon}$//


$|\frac{1}{n} - 0| \rightarrow \frac{1}{n} = n > \frac{1}{\epsilon} \rightarrow N(\epsilon) = [\frac{1}{\epsilon} + 1 > \frac{1}{\epsilon} \rightarrow n > N(\epsilon) > \frac{1}{\epsilon}$


2) Докажем, что у последовательности $a= \lim(-1)^n$.

От противного: при достаточно больших  n $|a - a_n| < \frac{1}{2}$ и $|a - a_{n + 1}| < \frac{1}{2} ->...$
\\
Если последовательность имеет предел, то он единственный. Докажем от противного: пусть $\lim a_n = a, \lim a_n = b, a \neq b.  | a- b | > \epsilon $.
Но по определению существует $\epsilon > 0$ 



























\end{document}