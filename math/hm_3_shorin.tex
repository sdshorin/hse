\documentclass[a4paper]{article}


\usepackage[T2A]{fontenc}
\usepackage[utf8]{inputenc}
\usepackage[english,russian]{babel}


\usepackage{amsmath, amsfonts, amssymb, amsthm, mathtools}

\author{Шорин Сергей, БКНАД211}
\title{Матанализ ДЗ 3}
\date{\today}

\begin{document}

\maketitle



\section*{2.a}
Исследовать на сходимость последовательность $ a_1 = 1, a_{n+1}  = 1 - \frac{1}{4a_n}$


\section*{2.б}
Исследовать на сходимость последовательность $ a_1 = 1, a_{n+1}  = \sqrt{5 + a_n}$

Заметим, что $a_n = \sqrt{5 + \sqrt{a_{n - 1}}}, a_{n+1} = \sqrt{5 + \sqrt{5 +\sqrt{a_{n - 1}}}}$ - то есть последовательность является возрастающей.

Докажем по индукции, что для любого n $|a_n| < 5$.

$$a_2 = \sqrt{6} < 5$$
$$a_{n + 1} = \sqrt{5 + a_n}$$
Так как $a_n$ меньше 5, заменим $a_n$ на 5.
$$a_{n + 1} <= \sqrt{5 + 5} = \sqrt{10} < 5$$
Следовательно, последовательность ограничена и возрастает - следовательно, у нее есть предел.

Перейдем к пределу в обеих частях равенства:
$$A = \sqrt{5 + A}, A > 0$$
$$A^2 - A - 5 = 0$$
$$A = \frac{1 \sqrt{21}}{2} = 2.7913$$
\subsection*{Ответ}
$$A = \frac{1 \sqrt{21}}{2} = 2.7913$$

\section*{2.в}
Исследовать на сходимость последовательность $ a_1 = \frac{1}{2}, a_{n+1}  = \frac{4}{3}a_n - a^2_n$





\newpage

\section{4.a}
Найти предел, используя определение: 
$\displaystyle{\lim_{x \to \infty}} \frac{n^2 + 3n - 1}{3n^2 - 2n + 2}$

$$|\frac{n^2 + 3n - 1}{3n^2 - 2n + 2} - \frac{1}{3}| < \epsilon $$

$$\frac{3n^2 + 9n - 3 - 3n^2 + 2n - 2 }{9n^2 - 6n + 6} < \epsilon $$
$$\frac{ 11n - n}{9n^2 + n^2} < \frac{ 11n - 5}{9n^2 - 6n + 6} < \epsilon $$

$$\frac{ 10n}{10n^2} < \epsilon $$
$$\frac{ 1}{n} < \epsilon $$
$$N = \frac{ 1}{\epsilon} + 1 $$

\section{4.б}
Найти предел, используя определение: 
$\displaystyle{\lim_{x \to \infty}} \frac{\log_an}{n}$

$$\frac{\log_an}{n} < \epsilon $$
$$ \log_an < \epsilon n $$

$$ \frac{\ln n}{\ln a} < \epsilon n $$

$$ 1 < \ln n < \epsilon n \ln a $$
$$ 1 < \epsilon n \ln a $$
$$ \frac{1}{\epsilon \ln a} <  n  $$
$$ \frac{1}{\epsilon \ln a} <  n  $$
$$ N = \frac{1}{\epsilon \ln a} + 1  $$

\section{4.в}
Найти предел, используя определение: 
$\displaystyle{\lim_{x \to \infty}} \sqrt[n]{n}$

$$\sqrt[n]{n} - 1 < \epsilon$$

$$\sqrt[n]{n}  < \epsilon + 1$$ 

$$ n  < (1 + \epsilon ) ^ n$$ 
Заметим, что правую часть неравенства можно разложить по биному Ньютона.

$$ n  < 1 + n\epsilon + \frac{n(n-1)}{2}\epsilon^2 + ... $$ 

(НЕ РЕШЕНО)

\section{1.а}
Найти предел $ \displaystyle{\lim_{x \to \infty}}(\sqrt{(n+1)(n+2)} - \sqrt{n(n-1)} $

Домножим на $\sqrt{(n+1)(n+2)} + \sqrt{n(n-1)}$

$$ \displaystyle{\lim_{x \to \infty}}\frac{(n+1)(n+2) - n(n-1)}{\sqrt{(n+1)(n+2)} + \sqrt{n(n-1)}} $$

$$ \displaystyle{\lim_{x \to \infty}}\frac{ n^2 + 3n + 2 - n^2 + n}{\sqrt{n^2 + 3n + 2} + \sqrt{n^2-n)}} $$

$$ \displaystyle{\lim_{x \to \infty}}\frac{ 4n + 2 }{\sqrt{n^2 + 3n + 2} + \sqrt{n^2-n)}} $$

$$ \displaystyle{\lim_{x \to \infty}}\frac{ 4 + \frac{2}{n} }{\sqrt{1 + \frac{3}{n} + \frac{2}{n^2}} + \sqrt{1-\frac{1}{n}}} $$

При $x \to \infty$ $\frac{1}{n} \to 0$

$$ \displaystyle{\lim_{x \to \infty}}\frac{ 4 + (0) }{\sqrt{1 + (0) + (0)} + \sqrt{1-(0)}} = \frac{4}{2} = 2 $$

\subsection*{Ответ: $\lim = 2$}
\end{document}
