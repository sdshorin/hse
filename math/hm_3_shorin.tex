\documentclass[a4paper]{article}


\usepackage[T2A]{fontenc}
\usepackage[utf8]{inputenc}
\usepackage[english,russian]{babel}


\usepackage{amsmath, amsfonts, amssymb, amsthm, mathtools}

\author{Шорин Сергей, БКНАД211}
\title{Матанализ ДЗ 3}
\date{\today}

\begin{document}

\maketitle



\section*{2.a}
Исследовать на сходимость последовательность $ a_1 = 1, a_{n+1}  = 1 - \frac{1}{4a_n}$

Предположим, что $ \frac{1}{2} <= a_n <= 1$. Для $a_1$ это верно.

Заметим, что на рассматриваемом промежутке минимум у функции  $ f(x) = 1 - \frac{1}{4x}$ находится в точке $x = \frac{1}{2}$ и равен $\frac{1}{2}$, а максимум находится в точке $x = 1$ и равен $\frac{3}{4}$, а на рассматриваемом промежутке функция монотонна. Возьмем полученные максимум и минимум и подставим в функцию как аргумент. Так как 
$\frac{1}{2} <= f(\frac{1}{2}) = \frac{1}{2} <= f(1) = \frac{3}{4} <= 1$, значения функции из отрезка $ \frac{1}{2} <= a_n <= 1$ будут попадать в этот же  отрезок: $ \frac{1}{2} <=  f(x) <= 1 : x \in (\frac{1}{2}, 1)$

Сравним $a_n$ и $a_{n+1}$ на промежутке $ \frac{1}{2} <=  a_n <= 1$:

$$ a_n - a_{n+1} \overset{?}{>} 0$$
$$ a_n - 1 + \frac{1}{4a_n} \overset{?}{>} 0,  a_n > 0$$
$$ 4a_n^2 - 4a_n + 1 \overset{?}{>} 0,  a_n > 0$$


$$x_{1, 2} = \frac{4 +- \sqrt{0}}{8}  $$
Это парабола, значения которой всегда неотрицательны. Таким образом, последовательность не возрастающая, ограниченная и, следовательно, у нее есть предел.

Заменим $a_n$ на предел $A$.
$$A  = 1 - \frac{1}{4A}$$
Это квадратное уравнение мы уже решали выше, $A = \frac{1}{2}$.

\subsection*{Ответ:}
Предел равен $\frac{1}{2}$.







\section*{2.б}
Исследовать на сходимость последовательность $ a_1 = 1, a_{n+1}  = \sqrt{5 + a_n}$

Заметим, что $a_n = \sqrt{5 + \sqrt{a_{n - 1}}}, a_{n+1} = \sqrt{5 + \sqrt{5 +\sqrt{a_{n - 1}}}}$ - то есть последовательность является возрастающей.

Докажем по индукции, что для любого n $|a_n| < 5$.

$$a_2 = \sqrt{6} < 5$$
$$a_{n + 1} = \sqrt{5 + a_n}$$
Так как $a_n$ меньше 5, заменим $a_n$ на 5.
$$a_{n + 1} <= \sqrt{5 + 5} = \sqrt{10} < 5$$
Следовательно, последовательность ограничена и возрастает - следовательно, у нее есть предел.

Перейдем к пределу в обеих частях равенства:
$$A = \sqrt{5 + A}, A > 0$$
$$A^2 - A - 5 = 0$$
$$A = \frac{1 \sqrt{21}}{2} = 2.7913$$
\subsection*{Ответ}
$$A = \frac{1 \sqrt{21}}{2} = 2.7913$$







\section*{2.в}
Исследовать на сходимость последовательность $ a_1 = \frac{1}{2}, a_{n+1}  = \frac{4}{3}a_n - a^2_n$


Предположим, что $ \frac{1}{3} <= a_n <= \frac{1}{2}$. Для $a_1$ это верно.

Посмотрим, в каких точках функция $f(x) = \frac{4}{3}x - x^2$ принимает значения вне диапазона $ \frac{1}{3} <= y <= \frac{1}{2}$
Эта парабола с максимумом у точке (0.(6), 0.(4))(точка перегиба), с корнями в точках 0 и $\frac{4}{3}$.
$f(\frac{1}{3}) = \frac{1}{3}$, $f(\frac{1}{2}) = \frac{5}{12} < \frac{1}{2}$ Таким образом, $ \frac{1}{3} <= \frac{4}{3}x - x^2 <= \frac{1}{2}$ при $ x \in [\frac{1}{3}, \frac{1}{2}]$. (на этом отрезке парабола монотонна)

Следовательно, последовательность ограничена (предположение оказалось верным).

Рассмотрим, в каком диапазоне последовательность не возрастает:

$$\frac{4}{3}a_n - a^2_n$$
$$a_n(\frac{4}{3} - a_n) <= a_n$$
$$a_n(\frac{1}{3} - a_n) <= 0$$
Это парабола, направленная вниз, с корнями в точках 0 и $\frac{1}{3}$. Следовательно, на промежутке $[\frac{1}{3}, \infty)$ последовательность не возрастает.


Так как последовательность ограничена и не возрастает, у нее есть предел.

Заменим $a_n$  на предел A.

$$A(\frac{4}{3} - A) = A$$

$$ A_1 = 0, A_2 = \frac{1}{3}, A_2 \in [\frac{1}{3}, \frac{1}{2}]$$
Следовательно, предел равен $\frac{1}{3}$.
\subsection*{Ответ:}
Предел равен $\frac{1}{3}$.

\end{document}
