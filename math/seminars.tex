\documentclass[a4paper]{article}


\usepackage[T2A]{fontenc}
\usepackage[utf8]{inputenc}
\usepackage[english,russian]{babel}


\usepackage{amsmath, amsfonts, amssymb, amsthm, mathtools}

\author{Sergey}
\title{Матанализ}
\date{\today}

\begin{document}

\maketitle

\newpage



задача 1
\[
2.13(9174) = x
213.(9174) = 100x
2139174.(9174) = 100000x
(вычитаем)
\]
1.b
разобьем дроби на пары, чтобы сумма пары была равна 0.(8888)
\[
0.(8) = x
8.8 = 10 x
8 = 9x
x = 8/9
Всего пар 120 (5!)
60 пар чисел
60*(8/9)
\]


2.a Задача на дом (как на лекции)

2.b доказать иррациональность числа\[ \sqrt{2} + \sqrt{3}
Докажем от противного
\sqrt{2} + \sqrt{3} = p/q
2+ 2\sqrt{6} + 3 = p^2/q^2
2\sqrt{6} =  p^2/q^2 - 5
(p^2/q^2 - 5 = p_1/q_1  ) 
2\sqrt{6} = p_1/q_1
\sqrt{6} = 2*p_1/q_1 = p_2/q_2
6 = p^2/q^2
6q^2 = p^2
p^2 делится на 6
p - четное, p = 2k
3q^2 = 2k^2 -> q четное
Следовательно, p/q сократимая дробь (что противоречит)
\]

2.
\[
/sin(\pi / 9)
\sin(3a) = 3\sin(a) - 4\sin^3(a)
\sin(\pi/9) = 3\sin(\pi/9) - 4\sin^3(\pi/9)
\sqrt{3}/2 = 3\sin(\pi/9)- \sin^3(\pi/9)
От противного: пусть \sin(\pi/9) \in Q -> \sin^3(\pi/9) -> \sin(\pi/9) \in Q -> \sqrt{3}/2 \in Q (противоречие)
\]

Корень из натурального числа либо число наутральное, либо число иррациональное (так же доказывается, от пртивного)




3.a 
$
S = 1 + q +  q^2 + ...
Sq = q + q^2 + ...
(q-1)S = (q^{n+1} -1)
S = (q^{n+1} -1) / (q-1)
a^{n+1} + b^{n+1} = 
$



4.б

Индукция: проверить, что верно для 1.
Можно ли вывести для 2?

Верно ли для 3, если верно для 2?

верно ли для n+1, еслb верно для n?

\[1^2 + 2^2 + .. n^2 = n*(n+1)(2n + 1)/6\]

Для 1:
... (верно)

Предположение инфдукции:
Пусть при k верно
\[
1+ 2^2 + ... k^2 = k(k+1)(2k+1)/6
1+ 2^2 + ... k^2 + (k+ 1)^2 = k(k+1)(2k+1)/6 + (k+ 1)^2 =  (k(k+1)(2k+1) + (k+ 1)^2)/6 = (k+1)(k+2)(2k+3)/6
\]


4.в
\[
(x + y)^n = \sum^n_{k=0}С_n^kx^ky^{n-k}\]


\section*{Семинар 2}

$$\left \{ \lim\limits_{n\rightarrow \infty }x_{n}= a \right \} \Leftrightarrow \forall \varepsilon > 0: \exists N_{\varepsilon }\in \mathbb{N}:\forall n \geq N_{\varepsilon }: \left | x_{n}-a \right |< \varepsilon$$


б)
$a_n = \frac{1}{1*3} + \frac{1}{3*5} + \frac{1}{5*7} + ... + \frac{1}{(2n-1)*(2n+1)} = \frac{1}{2}(1 - \frac{1}{2n + 1})$
$$| \frac{1}{2} + \frac{1}{4n + 2} - \frac{1}{2}|< \epsilon <=> 4n+ 2 > \frac{1}{\epsilon} <=> 4n > \frac{1}{\epsilon} - 2 <=> n > \frac{1}{4\epsilon} - \frac{1}{2} <=> N(\epsilon)  = [ \frac{1}{4\epsilon} - \frac{1}{2}] + 1$$




\end{document}